\documentclass[11pt,a4paper]{article}

\usepackage[slovene]{babel}
\usepackage[utf8x]{inputenc}
\usepackage{graphicx}
\usepackage{url}
\usepackage{pdfpages}

\pagestyle{plain}

\begin{document}
\title{Poročilo pri predmetu \\
Analiza podatkov s programom R}
\author{Teja Rupnik}
\maketitle

\section{Izbira teme}
V projektu sem analizirala podatke o dokončanih stanovanjih v sloveniji ter prebivalstvo v sloveniji. Vse podatke sem primerjala med regijami v Sloveniji v letih od 2008 do 2013. Podatke sem našla na spletni strani Statističnega urada republike Slovenije. Poiskala sem primerjavo o dokončanih stanovanjih po investitorjih in glede na nastanek, ter skupno število dokončanih gradenj po letih. Prav tako sem pridobila podatke o prebivalstvu po letih. Podatke sem pridobila tako v CSV kot v HTML obliki

Vse podatke o gradnji stanovanj sem dobila iz spletne strani Statističnega urada republike Slovenij:
\url{http://pxweb.stat.si/pxweb/Dialog/varval.asp?ma=1906902S&ti=&path=../Database/Ekonomsko/19_gradbenistvo/05_19069_graditev_stan/&lang=2}

Podatke o prebivalstvu v Sloveniji pa prav tako iz spletne strani Statističnega urada republike Slovenije:
\url{http://pxweb.stat.si/pxweb/Dialog/varval.asp?ma=05C2001S&ti=&path=../Database/Dem_soc/05_prebivalstvo/10_stevilo_preb/10_05C20_prebivalstvo_stat_regije/&lang=2}

Moji cilj projekt je analizirati kako je rast prebivalstva povezana gradno stanovanj ter primerajti kako se ti podatki razlikujejo po regijah.

\newpage
\section{Obdelava, uvoz in čiščenje podatkov}
V drugi fazi projekta sem uvozila podatke, ki sem jih našla v prvi fazi, kot razpredelnice. Uvozila sem 4 razpredelnice, vendar sem v nadaljno analizo vključila le dve razpredelnici in iz njiju narisala grafa.

Prvi graf prikazuje število dokončanih stanovanj. Uporabila sem podatke med leti 2008 in 2013 in jih prikazala za vse regije v Sloveniji.
\includepdf[pages={1}]{../slike/grafi.pdf}

Drug graf pa prikazuje prebivalstvo po regijah v Sloveniji. Za lažjo primerjavo sem uporabila podatke med leti 2008 in 2013.
\includepdf[pages={3}]{../slike/grafi.pdf}

\newpages
\section{Analiza in vizualizacija podatkov}

V tej fazi sem se odločila, da bom v svoj projekt vključila zemljevide, ki prikazujejo število zgrajenih stanovanj po regijah.

In sicer narisala sem graf, ki prikazuje povprečno število zgrajenih stanovanj po regijah med leti 2008 in 2013. Iz zemljevida lahko razberemo, da ima najvišje število zgrajenih stanovanj osrednja Slovenija. Najmanjše število zdrajenih stanovanj pa imajo regije na zahodu Slovenije in Koroška.

\makebox[\textwidth][c]{
\includegraphics[width=1.2\textwidth]{../slike/regije1.pdf}
}

Nato sem v treh grafih primerjala gradnjo stanovanj po regijah v letih 2008, 2011 ter 2013.

\makebox[\textwidth][c]{
\includegraphics[width=1.2\textwidth]{../slike/regije2.pdf}
}
\vspace{5mm}
\makebox[\textwidth][c]{
\includegraphics[width=1.2\textwidth]{../slike/regije3.pdf}
}
\vspace{5mm}
\makebox[\textwidth][c]{
\includegraphics[width=1.2\textwidth]{../slike/regije4.pdf}
}

Da bi lažje analizirala svoje podatke, sem se odločila, do bom v svoj projekt uvozila še poatke o rasti prepivalstav Sloveniji po regijah. Tudi te podatke sem uvozila kot csv iz spletne strani Statističnega urada Republike Slovenije. Nardeila sem tabelo:

\includepdf[pages=1]{../slike/grafi.pdf}

Tudi iz teh podatkov sem narisala graf, ki prikazuje povprečno število prebivalcev po regijah med leti 2008 in 2014. Tudi ta zemlevid pokaže, da je največ prebivalcev v osrednji Sloveniji.

\makebox[\textwidth][c]{
\includegraphics[width=1.2\textwidth]{../slike/prebivalstvo1.pdf}
}

Za dobro primerjavo sem tudi tukaj posebaj narisala zemljevide števila prebivalcev po regijah za leto 2008, 2011, 2013. Seveda sem izbrala enaka leta kot pri prikazu števila stanovanj.

\makebox[\textwidth][c]{
\includegraphics[width=1.2\textwidth]{../slike/prebivalstvo2.pdf}
}
\vspace{5mm}
\makebox[\textwidth][c]{
\includegraphics[width=1.2\textwidth]{../slike/prebivalstvo3.pdf}
}
\vspace{5mm}
\makebox[\textwidth][c]{
\includegraphics[width=1.2\textwidth]{../slike/prebivalstvo4.pdf}
}


%\section{Napredna analiza podatkov}

%\includegraphics{../slike/naselja.pdf}

\end{document}
