\documentclass[11pt,a4paper]{article}

\usepackage[slovene]{babel}
\usepackage[utf8x]{inputenc}
\usepackage{graphicx}
\usepackage{url}
\usepackage{pdfpages}

\pagestyle{plain}

\begin{document}
\title{Poročilo pri predmetu \\
Analiza podatkov s programom R}
\author{Teja Rupnik}
\maketitle

\section{Izbira teme}

V projektu, bom analizirala podatke o dokončanih stanovanjih. Primerjala bom dokončana stanovanja glede na nastanek, torej stanovanja pridobljena z novogradnjo, povečavo in spremembo namembnosti. Nato bom primerjala stanovanja po investitorjih, ki so investirali v gradnjo. Radelila jih bom na pravne in fizične osebe. Vse te podatke pa bom primerjala po regijah v Sloveniji.

Podatke za moj projekt sam dobila na spletni strani Statističnega urada Republike Slovenije. 
\url{http://pxweb.stat.si/pxweb/Dialog/varval.asp?ma=1906902S&ti=&path=../Database/Ekonomsko/19_gradbenistvo/05_19069_graditev_stan/&lang=2}

Moj cilj projekta je, da analiziram katere regije imajo največ dokončanih stanovanj, kateri investitorji so največ investirali v izgradnjo stanovanje in primerjati kako so stanovanja nastala.

\newpage
\section{Obdelava, uvoz in čiščenje podatkov}
V drugi fazi projekta sem uvozila dve razpredelnici v csv obliki in eno v html obliki. Vsi podatki za tabele so iz spletne strani SURSa.

Uvoz razpredelnic v csv obliki je bil nekoliko lažji prav tako izdelava grafov za prvi dve tabeli. Tretja tabela in graf pa st bila nekoliko težja, vendar mi je z nekoliko truda uspelo tudi to.

Tu so vsi moji grafi:

\includepdf[pages={1-2}]{../slike/grafi.pdf}

\section{Analiza in vizualizacija podatkov}

V tej fazi sem se odločila, da bom v svoj projekt vključila zemljevide, ki prikazujejo število zgrajenih stanovanj po regijah.

In sicer narisala sem graf, ki prikazuje povprečno število zgrajenih stanovanj po regijah med leti 2008 in 2013. Iz zemljevida lahko razberemo, da ima najvišje število zgrajenih stanovanj osrednja Slovenija. Najmanjše število zdrajenih stanovanj pa imajo regije na zahodu Slovenije in Koroška.

\makebox[\textwidth][c]{
\includegraphics[width=1.2\textwidth]{../slike/regije1.pdf}
}

Nato sem v treh grafih primerjala gradnjo stanovanj po regijah v letih 2008, 2011 ter 2013.

\makebox[\textwidth][c]{
\includegraphics[width=1.2\textwidth]{../slike/regije2.pdf}
}
\vspace{5mm}
\makebox[\textwidth][c]{
\includegraphics[width=1.2\textwidth]{../slike/regije3.pdf}
}
\vspace{5mm}
\makebox[\textwidth][c]{
\includegraphics[width=1.2\textwidth]{../slike/regije4.pdf}
}

Da bi lažje analizirala svoje podatke, sem se odločila, do bom v svoj projekt uvozila še poatke o rasti prepivalstav Sloveniji po regijah. Tudi te podatke sem uvozila kot csv iz spletne strani Statističnega urada Republike Slovenije. Nardeila sem tabelo:

\includepdf[pages=1]{../slike/grafi.pdf}

Tudi iz teh podatkov sem narisala graf, ki prikazuje povprečno število prebivalcev po regijah med leti 2008 in 2014. Tudi ta zemlevid pokaže, da je največ prebivalcev v osrednji Sloveniji.

\makebox[\textwidth][c]{
\includegraphics[width=1.2\textwidth]{../slike/prebivalstvo1.pdf}
}

Za dobro primerjavo sem tudi tukaj posebaj narisala zemljevide števila prebivalcev po regijah za leto 2008, 2011, 2013. Seveda sem izbrala enaka leta kot pri prikazu števila stanovanj.

\makebox[\textwidth][c]{
\includegraphics[width=1.2\textwidth]{../slike/prebivalstvo2.pdf}
}
\vspace{5mm}
\makebox[\textwidth][c]{
\includegraphics[width=1.2\textwidth]{../slike/prebivalstvo3.pdf}
}
\vspace{5mm}
\makebox[\textwidth][c]{
\includegraphics[width=1.2\textwidth]{../slike/prebivalstvo4.pdf}
}


%\section{Napredna analiza podatkov}

%\includegraphics{../slike/naselja.pdf}

\end{document}
