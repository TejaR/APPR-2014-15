\documentclass[11pt,a4paper]{article}

\usepackage[slovene]{babel}
\usepackage[utf8x]{inputenc}
\usepackage{graphicx}
\usepackage{url}
\usepackage{pdfpages}

\pagestyle{plain}

\begin{document}
\title{Poročilo pri predmetu \\
Analiza podatkov s programom R}
\vspace{15mm}
\textbf{\emph{Voda iz javnega vodovoda}}}
\author{Teja Rupnik}
\maketitle

\newpage
\section{Izbira teme}
V projektu sem analizirala podatke o dokončanih stanovanjih v sloveniji ter prebivalstvo v sloveniji. Vse podatke sem primerjala med regijami v Sloveniji v letih od 2008 do 2013. Podatke sem našla na spletni strani Statističnega urada republike Slovenije. Poiskala sem primerjavo o dokončanih stanovanjih po investitorjih in glede na nastanek, ter skupno število dokončanih gradenj po letih. Prav tako sem pridobila podatke o prebivalstvu po letih. Podatke sem pridobila tako v \verb|csv| kot v  \verb|html| obliki

Vse podatke o gradnji stanovanj sem dobila iz spletne strani Statističnega urada republike Slovenij:
\url{http://pxweb.stat.si/pxweb/Dialog/varval.asp?ma=1906902S&ti=&path=../Database/Ekonomsko/19_gradbenistvo/05_19069_graditev_stan/&lang=2}

Podatke o prebivalstvu v Sloveniji pa prav tako iz spletne strani Statističnega urada republike Slovenije:
\url{http://pxweb.stat.si/pxweb/Dialog/varval.asp?ma=05C2001S&ti=&path=../Database/Dem_soc/05_prebivalstvo/10_stevilo_preb/10_05C20_prebivalstvo_stat_regije/&lang=2}

Moji cilj projekt je analizirati:
\begin{itemize}
\item{Kako je rast prebivalstva povezana gradno stanovanj.} 
\item{Primerajti kako se ti podatki razlikujejo po regijah.}
\end{itemize}

\newpage
\section{Obdelava, uvoz in čiščenje podatkov}
V drugi fazi projekta sem uvozila podatke, ki sem jih našla v prvi fazi, kot razpredelnice. Uvozila sem 4 razpredelnice.

\begin{enumerate}
\item{tabela: Prikazuje Število izgrajenih stanovanje po regijah v Slovenije glede na investitorja (pravna oseba, fizična oseba) za leto 2013 ter povprečje za leto 2013. tevilo stanovanj [številske spremenljivke] je podana v stolpcih, v vrsticah pa imamo regije v Sloveniji [imenske spremenljivke].}
\item{tabela: Prikazuje tevilo izgrajenih stanovanj po regijah v Sloveniji, med leti 2008 in 1013. Leta [številske spremenljivke] so podana v stolpcu, v vrsticah pa imamo imena rgij v Sloveniji [imenske spremenljivke].}
\item{tabela: Prikazuje število izgrajenih stanovanje po regijah v Slovenije glede na način pridobitve (novogradnja, povečava, sprememba namembnosti) za leto 2013. Število stanovanj [številske spremenljivke] so podana v stolpcih, v vrsticah pa so podane regije v Sloveniji [imenske spremenljivke].}
\item{tabela: Prikazuje tisoče prebivalcev po regijah v Sloveniji med leti 2008 in 2013 ter povprečje vseh let. Leta [številske spremenljivke] so podana v stolpcih, v vrsticah pa so podane regije v Sloveniji [imenske spremenljivke].}
\end{enumerate}

Vendar sem v nadaljno analizo vključila le dve razpredelnici in iz njiju narisala grafa.

\newpage
\textbf{1.GRAF: \emph{Izgrajena stanovanja, med leti 2008 in 2013}}: Za prvo tabelo, ki govori o izgrajenih stanovanjih sem se odločila, da bom uporabila stolpični graf (barplot), saj sem želela, da bi iz grafa videli primerjavo tako med regijami kot tudi med posameznimi leti.\\
\textbf{Interpretacija}: Iz grafa lahko vidimo, da se je v preteklih letih največ stanovanj pridobilo v osrednji Sloveniji, ter da je pridobitev novi stanovanj povsod v Sloveniji upadala.

\makebox[\textwidth][c]{
\includepdf[pages=1]{../slike/grafi.pdf}
}

\newpage
\textbf{2.GRAF: \emph{Prebivalstvo, med leti 2008 in 2013}}: Tudi za drugo tabelo, ki prikazuje prebivalstvo po regijah med leti 2008 in 2012 se uporabila stolpični graf (barplot), saj sem prav tako želela, da bi iz grafa videli primerjavo tako med regijami kot tudi med posameznimi leti.\\
\textbf{Interpretacija}: Iz grafa lahko vidimo, da se je v največ prebivalstva prav v osrednji Sloveniji, ter da prebivalstvo povsot rahlo upadalo vendar nekoliko počasneje kot pri pridobivanju novi stanovanj.

\makebox[\textwidth][c]{
\includepdf[pages=3]{../slike/grafi.pdf}
}

\newpage
\section{Analiza in vizualizacija podatkov}

V tej fazi sem se odločila, da bom v svoj projekt vključila zemljevide, ki prikazujejo število zgrajenih stanovanj po regijah.

In sicer narisala sem graf, ki prikazuje povprečno število zgrajenih stanovanj po regijah med leti 2008 in 2013. Iz zemljevida lahko razberemo, da ima najvišje število zgrajenih stanovanj osrednja Slovenija. Najmanjše število zdrajenih stanovanj pa imajo regije na zahodu Slovenije in Koroška.

\makebox[\textwidth][c]{
\includegraphics[width=1.2\textwidth]{../slike/regije1.pdf}
}

Nato sem v treh grafih primerjala gradnjo stanovanj po regijah v letih 2008, 2011 ter 2013.

\makebox[\textwidth][c]{
\includegraphics[width=1.2\textwidth]{../slike/regije2.pdf}
}
\vspace{5mm}
\makebox[\textwidth][c]{
\includegraphics[width=1.2\textwidth]{../slike/regije3.pdf}
}
\vspace{5mm}
\makebox[\textwidth][c]{
\includegraphics[width=1.2\textwidth]{../slike/regije4.pdf}
}

Da bi lažje analizirala svoje podatke, sem se odločila, do bom v svoj projekt uvozila še poatke o rasti prepivalstav Sloveniji po regijah. Tudi te podatke sem uvozila kot csv iz spletne strani Statističnega urada Republike Slovenije. Nardeila sem tabelo:

\includepdf[pages=1]{../slike/grafi.pdf}

Tudi iz teh podatkov sem narisala graf, ki prikazuje povprečno število prebivalcev po regijah med leti 2008 in 2014. Tudi ta zemlevid pokaže, da je največ prebivalcev v osrednji Sloveniji.

\makebox[\textwidth][c]{
\includegraphics[width=1.2\textwidth]{../slike/prebivalstvo1.pdf}
}

Za dobro primerjavo sem tudi tukaj posebaj narisala zemljevide števila prebivalcev po regijah za leto 2008, 2011, 2013. Seveda sem izbrala enaka leta kot pri prikazu števila stanovanj.

\makebox[\textwidth][c]{
\includegraphics[width=1.2\textwidth]{../slike/prebivalstvo2.pdf}
}
\vspace{5mm}
\makebox[\textwidth][c]{
\includegraphics[width=1.2\textwidth]{../slike/prebivalstvo3.pdf}
}
\vspace{5mm}
\makebox[\textwidth][c]{
\includegraphics[width=1.2\textwidth]{../slike/prebivalstvo4.pdf}
}


%\section{Napredna analiza podatkov}

%\includegraphics{../slike/naselja.pdf}

\end{document}
