\documentclass[11pt,a4paper]{article}

\usepackage[slovene]{babel}
\usepackage[utf8x]{inputenc}
\usepackage{graphicx}
\usepackage{url}
\usepackage{pdfpages}

\pagestyle{plain}

\begin{document}

\begin{titlepage}
\newcommand{\HRule}{\rule{\linewidth}{0.5mm}}
\center

\textsc{\LARGE Fakulteta za matematiko in fiziko}\\[3 cm]
\textsc{\Large Poročilo pri predmetu}\\[0.5cm]
\textsc{\large Analiza podatkov s programom R}\\[2 cm]
{ \huge \bfseries Gradnja stanovanj v Sloveniji}\\[0.4cm]
\HRule \\[9 cm]


\begin{minipage}{0.4\textwidth}
\begin{flushleft} \large
\emph{Avtor:}\\
Teja \textsc{Rupnik}
\end{flushleft}
\end{minipage}
~
\begin{minipage}{0.4\textwidth}
\begin{flushright} \large
\emph{Mentor:} \\
Dr. Janoš \textsc{Vidali}
\end{flushright}
\end{minipage}\\[2 cm]

{\large \today}\\[3cm] 


\end{titlepage}
\pagebreak

\newpage
\section{Izbira teme}
V projektu sem analizirala podatke o dokončanih stanovanjih v Sloveniji ter podatke o prebivalstvu v Sloveniji. Vse podatke sem primerjala med regijami v Sloveniji v letih od 2008 do 2013. Podatke sem našla na spletni strani Statističnega urada republike Slovenije. Poiskala sem primerjavo o dokončanih stanovanjih po investitorjih in glede na nastanek, ter skupno število dokončanih gradenj po letih. Prav tako sem pridobila podatke o prebivalstvu po letih. Podatke sem pridobila tako v \large\textbf{csv} kot v  \large\textbf{html} obliki.
\newline

Vse podatke o gradnji stanovanj sem dobila iz spletne strani Statističnega urada republike Slovenij:\newline
\url{http://pxweb.stat.si/pxweb/Dialog/varval.asp?ma=1906902S&ti=&path=../Database/Ekonomsko/19_gradbenistvo/05_19069_graditev_stan/&lang=2}
\newline

Podatke o prebivalstvu v Sloveniji pa prav tako iz spletne strani Statističnega urada republike Slovenije:\newline
\url{http://pxweb.stat.si/pxweb/Dialog/varval.asp?ma=05C2001S&ti=&path=../Database/Dem_soc/05_prebivalstvo/10_stevilo_preb/10_05C20_prebivalstvo_stat_regije/&lang=2}
\newline

Moji cilj projekt je analizirati:
\begin{itemize}
\item{Kako je rast prebivalstva povezana z gradnjo stanovanj.} 
\item{Kako se je skozi čas spremijalo število stanovanj po regijah.} 
\item{Primerajti kako se ti podatki razlikujejo po regijah.}
\end{itemize}

\newpage
\section{Obdelava, uvoz in čiščenje podatkov}
V drugi fazi projekta sem uvozila podatke, ki sem jih našla v prvi fazi, kot razpredelnice. Uvozila sem 4 razpredelnice.
\newline

Uvožene podatke sem shtanila v mapo \verb|podatki| ter jih kot tabele uvozila v mapi \verb|uvozi| v skripti \verb|uvozi.r|, ki sem jo vključila v \verb|projekt.r|.

\begin{enumerate}
\item{\verb|tabela:| Prikazuje Število izgrajenih stanovanje po regijah v Slovenije glede na investitorja (pravna oseba, fizična oseba) za leto 2013 ter povprečje za leto 2013. Število stanovanj [številske spremenljivke] je podana v stolpcih, v vrsticah pa imamo regije v Sloveniji [imenske spremenljivke].}
\item{\verb|tabela:| Prikazuje število izgrajenih stanovanj po regijah v Sloveniji, med leti 2008 in 1013. Leta [številske spremenljivke] so podana v stolpcu, v vrsticah pa imamo imena regij v Sloveniji [imenske spremenljivke].}
\item{\verb|tabela:| Prikazuje število izgrajenih stanovanje po regijah v Slovenije glede na način pridobitve (novogradnja, povečava, sprememba namembnosti) za leto 2013. Število stanovanj [številske spremenljivke] so podana v stolpcih, v vrsticah pa so podane regije v Sloveniji [imenske spremenljivke].}
\item{\verb|tabela:| Prikazuje tisoče prebivalcev po regijah v Sloveniji med leti 2008 in 2013 ter povprečje vseh let. Leta [številske spremenljivke] so podana v stolpcih, v vrsticah pa so podane regije v Sloveniji [imenske spremenljivke].}
\end{enumerate}

Vendar sem v nadaljno analizo vključila le dve razpredelnici in iz njiju narisala grafa, ki se nahajata v mapi \verb|slike| v skripti \verb|Grafi.R|, ki je vkjlučena v \verb|projekt.r|.

\newpage
\textbf{1.GRAF: \emph{Gradnja stanovanj med leti 2008 in 2013}}: Za prvo tabelo, ki govori o izgrajenih stanovanjih sem se odločila, da bom uporabila stolpični graf (barplot), saj sem želela, da bi iz grafa videli primerjavo tako med regijami kot tudi primerjavo skozi leta.\\
\newline
\textbf{Interpretacija}: Iz grafa lahko vidimo, da se je v preteklih letih največ stanovanj pridobilo v osrednji Sloveniji, ter da je pridobitev novi stanovanj povsod v Sloveniji upadala.
\newline
\makebox[\textwidth][c]{
\includegraphics[width=12cm]{../slike/grafi1.pdf}
}

\newpage
\textbf{2.GRAF: \emph{Prebivalstvo med leti 2008 in 2013}}: Tudi za drugo tabelo, ki prikazuje prebivalstvo po regijah med leti 2008 in 2012 se uporabila stolpični graf (barplot), saj sem prav tako želela, da bi iz grafa videli primerjavo tako med regijami kot tudi primerjavo skozi leta.\\
\newline
\textbf{Interpretacija}: Iz grafa lahko vidimo, da je največ prebivalstva prav v osrednji Sloveniji, ter da je prebivalstvo povsod rahlo naraščalo, vendar nekoliko počasneje kot pri pridobivanju novi stanovanj.
\newline
\makebox[\textwidth][c]{
\includegraphics[width=12cm]{../slike/grafi3.pdf}
}

\newpage
\section{Analiza in vizualizacija podatkov}

V tretji fazi sem narilsala sklop zemljevidov, ki prikazujejo prebivalstvo po regijah v Sloveniji skozi čas, sklop zemljevidov, ki prikazujejo gradno stanovanj po regijah v Sloveniji skozi čas ter sklop zemljevidov, ki prikazuje gradnjo stanovanj na prebivalca po regijah v Sloveniji med leti 2008 in 2013.
\newline

Program za risanje zemljevidov sem napisala v skripto \verb|vizualizacija.r| v mapi \verb|vizualizacija| in ga vključila v \verb|projekt.r|
\newline

Naslednji zemljevidi prikazujejo \textbf{gradnjo stanovanj na prebivalca po regijah v Sloveniji za leta 2008, 2011 in 2013.} Primerjam lahko kako se je le-ta spreminjala skozi čas. Opazila sem, da je v večini regij gradnja na prebivalca upadla, le v Zasavskem in Savinjskem je bila konstantna ter nizka.

\newpage
\includegraphics[width=0.5\textwidth]{../slike/slike-analiza/stanovanja_na_prebivalca1.pdf}
\newline
\includegraphics[width=0.5\textwidth]{../slike/slike-analiza/stanovanja_na_prebivalca2.pdf}
\newline
\includegraphics[width=0.5\textwidth]{../slike/slike-analiza/stanovanja_na_prebivalca3.pdf}

\newpage
\section{Napredna analiza podatkov}

Povzetek ugotovitv iz prejšnjih faz:
\begin{enumerate} 
\item{Največ novo pridobljenih stanovanj je bilo v osrednji Sloveniji, sledi ji Podravska.}
\item{Izrazito najmanj stanovanj se je pridobilo na zasavskem.}
\item{Če pa pogledamo na slovenijo kot celoto, opazimo da je pridobitev stanovanj vseskozi upadala.}
\item{Opazila sem tudi, da se je prebivalstvo gibalo skladno s stanovanji, saj je prav tako prebivalstvo največje v osrednji Sloveniji, kateri sledi Posavska, najmanjša pa v Zasavskem.}
\end{enumerate}

Do sedaj sem torej ugotovila že kar nekaj zanimivih podatkov. Sedaj pa bi v zadnji fazi rada ugotovila kako se je spreminjalo število stanovanj na prebivalca v Sloveniji med leti 2008 in 2013. Zanimam me ali so splošnemu modelu v Sloveniji sledile vse posamezne regije.

\newpage
Tako sem naredila dodaten graf, ki prikazuje število stanovanj na prebivalca po regijah v Sloveniji med leti 2008 in 2013:
\newline

\textbf{3.GRAF:\emph{Stanovanja na prebivalca med leti 2008 in 2013}}\\
\newline
\makebox[\textwidth][c]{
\includegraphics[width=12cm]{../slike/slike-analiza/grafi_analiza1.pdf}
}

\newpage
Opažanja iz grafa:
\begin{enumerate}
\item{Če pogledam na graf kot celoto opažam, da gradnja stanovanj na prebivalca upadala.}
\item{Na tem grafu pa z največjim številom stanovanj na prebivalca izstopata Obalno Kraška in Notranjo Kraška regija.}
\item{Leta 2012 opazim odstopanje Obalno Kraške regije s skoraj dvakrat toliko stanovanj na prebivalca kot ostale regije.}
\item{Leta 2008 ima dokaj veliko stanovanj na prebivalca tudi Podravska regija. Podravska regija je bila visoko tudi v grafih o stanovanjih in prebivalstvu posebaj.}
\end{enumerate}

\newpage
Na prejšnjem grafu sem ocenila, da je celotno gledano v Slovenji število stanovanj na prebivalca upadalo. Sedaj bom to prikazala z grafom in videla, če ta trditev drži.
\newline

\textbf{4.GRAF:\emph{Stanovanja na prebivalca v Sloveniji med leti 2008 in 2013}}\\
\newline
\makebox[\textwidth][c]{
\includegraphics[width=12cm]{../slike/slike-analiza/grafi_analiza2.pdf}
}

\newpage
\textbf{Sklepi:}
\begin{enumerate}
\item{Graf prikazuje upadanje stanovanj na prebivalca v sloveiji.}
\item{Temu modelu z manjšimi odstopanji sledi večino regij v Sloveniji.}
\item{Največje odstopanje se pojavi v Obalno Kraška regija, kjer v leto 2012 opazimo večji poskok v številu stanovanj na prebivalca.}
\end{enumerate}

\textbf{Napoved za prihodnost:}\newline
Iz rezultatov, ki sem jih z svojim projektom dobila sklepam, da mo število stanovanj na prebivalca še naprej padala. Ker pa je iz graf mogoče razbrati padajočo logaritemsko funkcijo sklepam, da bo upad stanovanj na prebivalca vedno manjši in se bo s časom ustali in ostal konstanten. Predvidevam, da je morda vzrok temu upadanju gospodarska kriza, ki se je pojavila ravno leta 2008, vendar bi za to analizo potrebovala še podatke od prejšnjih let. 

\newpage
\begin{thebibliography}{9}
\bibitem{1}
  \url{http://pxweb.stat.si/pxweb/Dialog/varval.asp?ma=1906902S&ti=&path=../Database/Ekonomsko/19_gradbenistvo/05_19069_graditev_stan/&lang=2}\\
  {Accessed: 01-03-2015}
\bibitem{2}
  \url{http://pxweb.stat.si/pxweb/Dialog/varval.asp?ma=05C2001S&ti=&path=../Database/Dem_soc/05_prebivalstvo/10_stevilo_preb/10_05C20_prebivalstvo_stat_regije/&lang=2}\\
  {Accessed: 01-03-2015}
\end{thebibliography}



\end{document}
