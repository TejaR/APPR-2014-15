\documentclass[11pt,a4paper]{article}

\usepackage[slovene]{babel}
\usepackage[utf8x]{inputenc}
\usepackage{graphicx}

\pagestyle{plain}

\begin{document}
\title{Poročilo pri predmetu \\
Analiza podatkov s programom R}
\author{Teja Rupnik}
\maketitle

\section{Izbira teme}

V projektu, bom analizirala podatke o dokončanih stanovanjih. Primerjala bom dokončana stanovanja glede na nastanek, torej stanovanja pridobljena z novogradnjo, povečavo in spremembo namembnosti. Nato bom primerjala stanovanja po investitorjih, ki so investirali v gradnjo. Radelila jih bom na pravne in fizične osebe. Vse te podatke pa bom primerjala po regijah v Sloveniji.

Podatke za moj projekt sam dobila na spletni strani Statističnega urada Republike Slovenije. (http://pxweb.stat.si/pxweb/Dialog/varval.asp?ma=1906902S&ti=&path=../Database/Ekonomsko/19_gradbenistvo/05_19069_graditev_stan/&lang=2)

Moj cilj projekta je, da analiziram katere regije imajo največ dokončanih stanovanj, kateri investitorji so največ investirali v izgradnjo stanovanje in primerjati kako so stanovanja nastala.

\newpage
\section{Obdelava, uvoz in čiščenje podatkov}
V drugi fazi projekta sem uvozila dve razpredelnici v csv obliki in eno v html obliki. Vsi podatki za tabele so iz spletne strani SURSa.

Uvoz razpredelnic v csv obliki je bil nekoliko lažji prav tako izdelava grafov za prvi dve tabeli. Tretja tabela in graf pa st bila nekoliko težja, vendar mi je z nekoliko truda uspelo tudi to.

Tu so vsi moji grafi:

\includepdf[pages=(1-6)]{../slike/grafi.pdf}
%\section{Analiza in vizualizacija podatkov}

%\includegraphics{../slike/povprecna_druzina.pdf}

%\section{Napredna analiza podatkov}

%\includegraphics{../slike/naselja.pdf}

\end{document}
